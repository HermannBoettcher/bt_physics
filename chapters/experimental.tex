\documentclass[../thesis.tex]{subfiles}

\begin{document}

  \chapter{Experimental}
  \label{ch:experimental}

    \section{Membrane production}
    \label{sec:membrane-production}

      The alumina membrane production process starts with a circular wafer of amorphous aluminum of $\SI{99,999}{\percent}$ purity of $\SI{1}{\milli\meter}$ thickness. First, the aluminum is anodized to create parallel pores that are arranged in a hexagonal order (compare \cref{subsec:anodizing}). For the conducted experiment, pore lengths $l_\mathrm{pore}=\SIrange{30}{60}{\micro\meter}$ are used. After the anodizing, the remaining aluminum of the wafer is dissolved by immersion in an acid as specified in \cref{eq:aluminum-dis-acid} in \cref{subsec:al-dissolution}). Last, the so called \textit{barrier layer} closing the pores on the bottom end is etched by floating the wafer on oxalic acid (\cref{subsec:barrier-layer-dissolution}).

      With an initial diameter of $d_\mathrm{wafer}=\SI{5}{\centi\meter}$, one wafer is cut into twelve square membranes of side length $l_\mathrm{membrane}=\SI{1}{\centi\meter}$. For 




\end{document}
