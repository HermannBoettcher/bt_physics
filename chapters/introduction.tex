\documentclass[../thesis.tex]{subfiles}

\begin{document}

  \chapter{Introduction}
  \label{ch:introduction}

    The team I conducted my internship in studies condensation and evaporation of helium in different confined materials. During the last 18 months the group focused on nanoporous membranes as a model system of pores of well defined diameters. The goal of these experiments is to measure the dependency of the pressure, at which the condensation and evaporation occur within the pores, on the pore diameters and the temperature relative to the critical temperature and to compare the results to theoretical models. In this respect, the advantage of helium is the ease of changing the temperature in a large range in respect to the critical temperature $T_\mathrm{c}$ from $\frac{T_\mathrm{c}}{4}$ to $T_\mathrm{c}$ in the setup.
    \medskip

    Nanoporous alumina membranes are widely used for ??? (ref). Typically they have a thickness of several tens of micrometers with pores aligned orthogonally to the membrane's surface. At \textit{Institut Néel}, they are synthesized by Laurent Cagnon using an anodization process. The typical pore diameters ranges from $\SI{40}{\nano\meter}$ to $\SI{60}{\nano\meter}$ with an inter pore distance of $\SI{100}{\nano\meter}$ to $\SI{120}{\nano\meter}$. Smaller diameters are then obtained by using atomic layer deposition (ALD) of alumina.
    \medskip

    At \textit{Institut Néel}, the observation of condensation and evaporation makes use of macroscopic techniques which are volumetric isotherm and light scattering measurements. Hereby, the pressure at which condensation and evaporation occurs is averaged over a very large number of pores. Determining the pore diameters from these quantities therefore requires the production of membranes that are as monodisperse as possible.
    \medskip

    The used membranes are characterized by two methods. First, absorption and desorption experiments using hexane as a fluid are conducted. The advantage of hexane over helium is that it allows experimenting at room temperature which makes for much faster executable experimentation. Second, reflection electron microscopy (REM) of the pore ends and the membranes' cross sections.
    \medskip

    When the author of this text arrived at \textit{Institut Néel} at the beginning of his internship, experiments with several membranes had already revealed that condensation and evaporation occured over a relatively broad range of pressures revealing a dispersion of pore diameters. This dispersion could be attributed to multiple effects which are the diameter variation along the length of a single pore (due to funnellization and/ or corrugations) and the distribution of the pores' diameters.
    \medskip

    The goals of the internship were then the following:
    \begin{itemize}
      \item Improving and systemizing the evaluation of the recorded isotherm data
      \item Performing isotherm measurements on many membranes as to confirm the previously reached conclusions statistically
      \item Comparing the pore diameters extracted from the volumetric measurements to the ones extracted from the REM results
      \item Improving the fabrication process to reduce the dispersion
      \item Testing the efficiency of the ALD process as a means to reduce the pore diameters.
    \end{itemize}

    \medskip

    Outlines to come..

    In the following,


\end{document}
