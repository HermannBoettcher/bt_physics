\documentclass[../../thesis.tex]{subfiles}


\begin{document}

\begin{figure}[ht]
    \centering
    \subfloat[]{
    \begin{tikzpicture}
        \pgfdeclarelayer{bg}    % declare background layer
        \pgfsetlayers{bg,main}  % set the order of the layers (main is the standard layer)
        \definecolor{copper}{rgb}{0.72, 0.45, 0.2}
        \def\holerad{0.3}
        \def\factor{4}
        \def\xradius{2}
        \def\yradius{2/\factor}
        \def\height{1.05cm}
        \def\xandy{2 and 2/\factor}
        \tikzset{
          pics/.cd, %
          disc/.style ={
            code = {
              %% the foundation
              \path [fill=black!15] (-\xradius,0) -- (-\xradius,-\height) arc
              (180:360:\xandy) -- (\xradius,0) arc (0:180:\xandy);%
              \path [top color=black!25, bottom color=white, opacity=0.2] (0,0) ellipse
              [x radius=\xradius, y radius =\yradius];%
              \path [left color=black!25, right color=black!15] (-\xradius,0) --
              (-\xradius,-\height) arc (180:240:\xandy) -- +(0,\height) arc
              (240:180:\xandy);%
              \path [left color=black!15, right color=black!30] (\xradius,0) --
              (\xradius,-\height) arc (360:320:\xandy) -- +(0,\height) arc
              (320:360:\xandy);
              %% rays in front
              \foreach \col/\r/\shift/\stop/\opacity in {%
                black/205/25/20/100, %
                black/295/35/30/100, %
                black/295/30/30/200, %
                black/295/25/20/300, %
                white/245/14/14/100, %
                white/245/12/12/20, %
                white/245/10/10/10} {%
                \foreach \i [evaluate={\opposite=\r-180;}] in {0,1,...,\stop}{%
                  \fill [\col, fill opacity = 1/\opacity] (\opposite:0.1 and
                  0.1/\factor) -- (\r+\shift-\i:\xandy) -- ++(0,-\height) arc
                  (\r+\shift-\i:\r-\shift+\i:\xandy) -- +(0,\height) -- cycle; }}
              %% rays in back
              \foreach \r/\shift/\stop/\opacity in {%
                25/25/20/100, %
                115/35/3/150,%
                115/30/23/100} {%
                \foreach \i [evaluate={\opposite=\r-180;}] in {0,1,...,\stop}{%
                  \fill [black, fill opacity = 1/\opacity] (\opposite:0.1 and 0.1/\factor) --
                  (\r+\shift-\i:\xandy) arc (\r+\shift-\i:\r-\shift+\i:\xandy) --
                  cycle; }}
              %% masking the four edges in the center
              \foreach \i in {0.1, 0.2, ..., 0.4}%
              \fill[black!15, opacity=0.7, path fading=fade out]
              (0,0) ellipse[x radius=\i, y radius =\i/\factor];
              %% the light and the dark arcs
              \foreach \i [evaluate={\start=185+10*\i; \finish=355-10*\i;}]%
              in {0.1, 0.2, ..., 1.5}{%
                \draw[white, opacity=0.04, line width=\i, yshift=0.02cm]
                (\start:\xandy) arc (\start:\finish:\xandy);
                \draw[black!80, opacity=0.05, line width=\i, yshift=-\height]
                (\start:\xandy) arc (\start:\finish:\xandy); }
            }
          },%
          disc bottom/.style = {
            code = {
              \foreach \i/\opacity in {%
                1/20,2/20,3/20,4/30,5/35,6/40,7/60,8/80,9/100,10/100,11/100,12/100}%
                \fill [red, fill opacity = 1/\opacity, yshift=-0.03cm] (0,-\height)
                ellipse [x radius = \xradius+\i/40, y radius = \yradius+\i/20/\factor];
              \path pic {disc};
            }
          },%
        }
        %
        \def\l{5.62};
        \def\h{2.5};
        \def\d{0.9};
        % procces simages
            \path (3,3.5) pic {disc bottom};
              \foreach \x/\y in {1.7/2, 2.5/1.7, 3.5/1.7, 4.3/2, 2.5/2.3, 3.5/2.3, 3/2}{
                    \begin{scope}[yshift = 1.5cm]
                      \draw[color=gray](\x,\y)ellipse[x radius=\holerad,y radius=\holerad/\factor];
                      \clip(\x,\y)ellipse[x radius=\holerad,y radius=\holerad/\factor];
                      \shade[left color=black,right color=black,middle color=white](\x-1,\y-1)rectangle(\x+1,\y+1);
                      \draw(\x,\y)ellipse[x radius=\holerad,y radius=\holerad/\factor];
                    \end{scope}}
            \draw (0.5,2.5) -- (0.5,4.5) node[anchor=south west] {Phosphoric acid};
            \draw (4,3.8) -- (4,4.5) node[anchor=south] {Wafer};
            \fill[blue!40, fill opacity = 0.3] (0,3) -- (0,0) -- (6,0) -- (6,3) -- (5,3) arc (360:180:2 and 0.5) -- cycle;
            \draw[line width = 2mm, color = gray] (0,5) -- (0,0) -- (6,0) -- (6,5);
    \end{tikzpicture}
    \label{fig:float}}
    \hfill
    \subfloat[]{
    \begin{tikzpicture}
        \pgfdeclarelayer{bg}    % declare background layer
        \pgfsetlayers{bg,main}  % set the order of the layers (main is the standard layer)
        \definecolor{copper}{rgb}{0.72, 0.45, 0.2}
        \def\holerad{0.3}
        \def\factor{4}
        \def\xradius{2}
        \def\yradius{2/\factor}
        \def\height{1.05cm}
        \def\xandy{2 and 2/\factor }
        \tikzset{
          pics/.cd, %
          disc/.style ={
            code = {
              %% the foundation
              \path [fill=black!15] (-\xradius,0) -- (-\xradius,-\height) arc
              (180:360:\xandy) -- (\xradius,0) arc (0:180:\xandy);%
              \path [top color=black!25, bottom color=white, opacity=0.2] (0,0) ellipse
              [x radius=\xradius, y radius =\yradius];%
              \path [left color=black!25, right color=black!15] (-\xradius,0) --
              (-\xradius,-\height) arc (180:240:\xandy) -- +(0,\height) arc
              (240:180:\xandy);%
              \path [left color=black!15, right color=black!30] (\xradius,0) --
              (\xradius,-\height) arc (360:320:\xandy) -- +(0,\height) arc
              (320:360:\xandy);
              %% rays in front
              \foreach \col/\r/\shift/\stop/\opacity in {%
                black/205/25/20/100, %
                black/295/35/30/100, %
                black/295/30/30/200, %
                black/295/25/20/300, %
                white/245/14/14/100, %
                white/245/12/12/20, %
                white/245/10/10/10} {%
                \foreach \i [evaluate={\opposite=\r-180;}] in {0,1,...,\stop}{%
                  \fill [\col, fill opacity = 1/\opacity] (\opposite:0.1 and
                  0.1/\factor) -- (\r+\shift-\i:\xandy) -- ++(0,-\height) arc
                  (\r+\shift-\i:\r-\shift+\i:\xandy) -- +(0,\height) -- cycle; }}
              %% rays in back
              \foreach \r/\shift/\stop/\opacity in {%
                25/25/20/100, %
                115/35/3/150,%
                115/30/23/100} {%
                \foreach \i [evaluate={\opposite=\r-180;}] in {0,1,...,\stop}{%
                  \fill [black, fill opacity = 1/\opacity] (\opposite:0.1 and 0.1/\factor) --
                  (\r+\shift-\i:\xandy) arc (\r+\shift-\i:\r-\shift+\i:\xandy) --
                  cycle; }}
              %% masking the four edges in the center
              \foreach \i in {0.1, 0.2, ..., 0.4}%
              \fill[black!15, opacity=0.7, path fading=fade out]
              (0,0) ellipse[x radius=\i, y radius =\i/\factor];
              %% the light and the dark arcs
              \foreach \i [evaluate={\start=185+10*\i; \finish=355-10*\i;}]%
              in {0.1, 0.2, ..., 1.5}{%
                \draw[white, opacity=0.04, line width=\i, yshift=0.02cm]
                (\start:\xandy) arc (\start:\finish:\xandy);
                \draw[black!80, opacity=0.05, line width=\i, yshift=-\height]
                (\start:\xandy) arc (\start:\finish:\xandy); }
            }
          },%
          disc bottom/.style = {
            code = {
              \foreach \i/\opacity in {%
                1/20,2/20,3/20,4/30,5/35,6/40,7/60,8/80,9/100,10/100,11/100,12/100}%
                \fill [red, fill opacity = 1/\opacity, yshift=-0.03cm] (0,-\height)
                ellipse [x radius = \xradius+\i/40, y radius = \yradius+\i/20/\factor];
              \path pic {disc};
            }
          },%
        }
        %
        \def\l{5.62};
        \def\h{2.5};
        \def\d{0.9};
        % procces simages
            \path (3,2.5) pic {disc bottom};
              \foreach \x/\y in {1.7/2, 2.5/1.7, 3.5/1.7, 4.3/2, 2.5/2.3, 3.5/2.3, 3/2}{
                    \begin{scope}[yshift = .5cm]
                      \draw[color=gray](\x,\y)ellipse[x radius=\holerad,y radius=\holerad/\factor];
                      \clip(\x,\y)ellipse[x radius=\holerad,y radius=\holerad/\factor];
                      \shade[left color=black,right color=black,middle color=white](\x-1,\y-1)rectangle(\x+1,\y+1);
                      \draw(\x,\y)ellipse[x radius=\holerad,y radius=\holerad/\factor];
                    \end{scope}}
            \draw (1.5,3.5) -- (1.5,4.5) node[anchor=south] {Phosphoric acid};
            \draw (4,2.7) -- (4,4.5) node[anchor=south] {Wafer};
            \fill[blue!40, fill opacity = 0.3] (0,4) -- (0,0) -- (6,0) -- (6,4) -- cycle;
            \draw[line width = 2mm, color = gray] (0,5) -- (0,0) -- (6,0) -- (6,5);
    \end{tikzpicture}
    \label{fig:immerse}}
    \caption{Setup for the \textit{barrier layer} dissolution. The wafer is floated \protect\subref{fig:float} or immersed \protect\subref{fig:immerse} in phosphoric acid dissolving alumina according to section ???. For best results a combination of floating and immersing is used (please refer to section \ref{sssec:barrier-layer-dissolution} for calrification of \textit{best}).}
    \label{fig:barrier-layer-dissolution-setup}
\end{figure}

\end{document}
