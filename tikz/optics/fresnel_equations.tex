\documentclass[../../thesis.tex]{subfiles}

\begin{document}
\begin{figure}[ht]
    \centering
    \begin{tikzpicture}
        \tikzstyle arrowstyle=[scale=2]
        \tikzstyle directed=[postaction={decorate,decoration={markings,
            mark=at position .65 with {\arrow[arrowstyle]{stealth}}}}]
        \pgfplotsset{trig format plots=rad}
        \pgfdeclarelayer{bg}    % declare background layer
        \pgfdeclarelayer{bbg}    % declare backbackground layer
        \pgfsetlayers{bg,main}  % set the order of the layers (main is the standard layer)
        \fill[blue!10] (6,0) -- (6,6) -- (12,6) -- (12,0) -- cycle;
        \draw[dashed] (0,2) -- (12,2);
        \draw (6,0) -- (6,6);
        \draw[color=red] (0,6) -- (6,2);
        \draw[color=red] (6,2) -- (3,0);
        \draw[color=red] (6,2) -- (12,0);
        % incident wave
        \draw[color=blue, ->] (3,4) -- (4,{4-2/3}) node[anchor=south, color=black] {$\mathbf{k}_\mathrm{i}$};
        \draw[circle] (3,4) circle (0.2);
        \draw[color=blue, ->] (3,4) -- (2.5,{4-2/3}) node[anchor=east, color=black] {$\mathbf{E}_\mathrm{i}$};
        \fill[circle, black] (3,4) circle (0.05) node[anchor=south,color=black] {$\mathbf{H}_\mathrm{i}$};
        % reflected wave
        \draw[color=blue, ->] (4,{2-4/3}) -- (3,{2-6/3}) node[anchor=south, color=black] {$\mathbf{k}_\mathrm{r}$};
        \draw[circle] (4,{2-4/3}) circle (0.2);
        \draw[color=blue, ->] (4,{2-4/3}) -- (3.5,{2-4/3+2/3}) node[anchor=east, color=black] {$\mathbf{E}_\mathrm{r}$};
        \fill[circle, black] (4,{2-4/3}) circle (0.05) node[anchor=north,color=black] {$\mathbf{H}_\mathrm{r}$};
        \draw[thick] ((4-0.15,{2-4/3-0.15}) -- ((4+0.15,{2-4/3+0.15});
        \draw[thick] ((4-0.15,{2-4/3+0.15}) -- ((4+0.15,{2-4/3-0.15});
        % transmitted wave
        \def\alpha{atan(1/3)};
        \draw[color=blue, ->] (9,{1}) -- ({9+cos(\alpha)*sqrt(13/9))},{1-sin(\alpha)*sqrt(13/9)}) node[anchor=south, color=black] {$\mathbf{k}_\mathrm{t}$};
        \draw[circle] (9,1) circle (0.2);
        \draw[color=blue, ->] (9,1) -- ({9-sin(\alpha)*(5/6)},{1-cos(\alpha)*(5/6)}) node[anchor=east, color=black] {$\mathbf{E}_\mathrm{t}$};
        \fill[circle, black] (9,1) circle (0.05) node[anchor=south,color=black] {$\mathbf{H}_\mathrm{t}$};
        %
        \coordinate (a) at (0,2);
        \coordinate (b) at (6,2);
        \coordinate (c) at (3,0);
        \pic[draw," $ \theta_\mathrm{r} $ ", draw=orange, <->, angle eccentricity=1.2, angle radius=1.5cm] { angle=a--b--c } ;
        \coordinate (a) at (0,6);
        \coordinate (b) at (6,2);
        \coordinate (c) at (0,2);
        \pic[draw," $ \theta_\mathrm{i} $ ", draw=orange, <->, angle eccentricity=1.2, angle radius=1.5cm] { angle=a--b--c } ;
        \coordinate (a) at (12,0);
        \coordinate (b) at (6,2);
        \coordinate (c) at (12,2);
        \pic[draw," $ \theta_\mathrm{t} $ ", draw=orange, <->, angle eccentricity=1.2, angle radius=1.5cm] { angle=a--b--c } ;
        \node at (5.5,5.5) {$n_1$};
        \node at (6.5,5.5) {$n_2$};
    \end{tikzpicture}
    \caption{Sketch for the derivation of the \textsc{Fresnel} equations. The indix $\mathrm{i}$ stands for the incident wave, $\mathrm{r}$ for the reflected and $\mathrm{t}$ for the transmitted. The continuity of the components parallel to the $n_1$ - $n_2$-interface is visualized. For the derivation please see \ref{sec:fresnel-equations}.}
    \label{fig:fresnel-equations}
\end{figure}


\end{document}
