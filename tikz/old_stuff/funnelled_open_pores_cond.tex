\documentclass[border = 3mm,
               tikz]{standalone}
\usepackage{tikz}
\usepackage{siunitx}
\usepackage{forest}
\usepackage{pstricks-add}
\usepackage{cleveref}\usepackage{relsize}

\usetikzlibrary{arrows,shapes,positioning}
\usetikzlibrary{decorations.markings}

\tikzset{fontscale/.style = {font=\relsize{#1}}
    }
\usetikzlibrary{arrows.meta,shapes,positioning}
\tikzset{
    full/.style={circle,draw,inner sep=0, minimum size=1mm,fill=black},
    every node/.style={minimum height=5mm,font=\footnotesize}
}

\begin{document}

    \pgfdeclarelayer{bg}    % declare background layer
    \pgfdeclarelayer{bbg}    % declare backbackground layer
    \pgfsetlayers{bbg,bg,main}  % set the order of the layers (main is the standard layer)
    \begin{tikzpicture}[every node/.style={fontscale = 3},
                        pore/.style = {line width = 0.2cm, color = gray},
                        film/.style = {line width = 0.15cm, color = blue!40},
                        thick_film/.style = {line width = 0.3cm, color = blue!40},
                        graph_label/.style={color = gray, line width = 0.07cm},
                        pore_collapse/.style = {color = blue!50, line width = 0.07cm, ->},
                        my_arrow/.style = {fontscale = 0.5}]
        \tikzstyle arrowstyle=[scale=2]
        \tikzstyle directed=[postaction={decorate,decoration={markings,
            mark=at position .65 with {\arrow[arrowstyle]{stealth}}}}]
        \foreach \Xcoor in {0, 5, 11, 16, 22}
        \draw[pore] (\Xcoor,-1) -- (\Xcoor + 0.3,-8);
        \foreach \Xcoor in {3, 8, 14, 19, 25}
        \draw[pore] (\Xcoor,-1) -- (\Xcoor -  0.3 ,-8);
        \node[gray] at (1.5,-0.6) {$r$};
        \draw[graph_label, ->,] (2,-0.6) -- (2.9,-0.6);
        \draw[graph_label, ->,] (1,-0.6) -- (0.1,-0.6);
        \node[gray] at (1.5,-8.4) {$r'$};
        \draw[graph_label, ->,] (2,-8.4) -- (2.5,-8.4);
        \draw[graph_label, ->,] (1,-8.4) -- (0.5,-8.4);
        \node[draw = yellow!90, fill = yellow!60, single arrow, my_arrow] at (4, -4.5) {\phantom{arrow}};
        \node[draw = green!70, fill = green!50, single arrow, my_arrow] at (9, -4.5) {\phantom{arrow}};
        \node[draw = red!70, fill = red!50, single arrow, my_arrow] at (15, -4.5) {\phantom{arrow}};
        \node[draw = blue!70, fill = blue!50, single arrow, my_arrow] at (21, -4.5) {\phantom{arrow}};
        \draw[color = red!50, line width = 0.1cm] (10,0) -- (20,0) -- (20,-9) -- (10,-9) -- cycle;
        \draw[color = red!60, line width = 0.1cm] (10,0) -- (20,0) -- (20,-9) -- (10,-9) -- cycle; \node[color = red!60] at (15,-10){spinodal pressure};
        \node[color = gray] at (6.5,-9) {cylindrical meniscus};
        \begin{pgfonlayer}{bg}    % select the background layer
            %filmed pore
            \draw[film] (5.14,-1) -- (5.14 + 0.3,-8);
            \draw[film] (7.86,-1) -- (7.86 - 0.3,-8);
            %thick filmed pore
            \draw[thick_film] (11.24,-1) -- (11.24 + 0.3,-8);
            \draw[thick_film] (13.76,-1) -- (13.76 - 0.3,-8);
            \begin{pgfonlayer}{bbg}
                \draw[pore_collapse] (11.3,-7.9) -- (12.2,-7.9);
                \draw[pore_collapse] (13.70,-7.9) -- (12.8,-7.9);
                \draw[pore_collapse] (12.5,-7.6) -- (12.5,-2);
            \end{pgfonlayer}
            %spinodal pore 
            \begin{pgfonlayer}{bbg}
                \draw[thick_film] (16.24,-1) -- (16.24 + 0.3,-8);
                \draw[thick_film] (18.76,-1) -- (18.76 - 0.3,-8);
                \fill[blue!40] (16.6,-8) --  (16,-1) -- (19,-1) -- (18.4,-8) --cycle;
                \path [name path = upper_meniscus, draw = none, fill = white](16.4,-0.9) arc[start angle = -180, end angle = 0, radius=1.1];
                \path [name path = lower_meniscus, draw = none, fill = white](16.4 + 0.3,-8.1) arc[start angle = 180, end angle = 0, radius=0.8];
            \end{pgfonlayer}
            %filled pore
            \fill[blue!40] (22 + 0.3,-8) -- (25 - 0.3,-8) -- (25,-1) -- (22,-1) -- cycle;
        \end{pgfonlayer};
    \draw[line width = 0.07cm,->, color = gray] (3,-19) -- (3,-12) node[anchor=south] {Liquid fraction};
    \draw[line width = 0.07cm,->, color = gray] (3,-19) -- (21,-19) node[anchor=west] {Pressure};
    %x labels
    \draw[graph_label] (3,-19) -- (3,-19.5) node[anchor=north] {$0$};
    \draw[graph_label] (10,-19) -- (10,-19.2) node[anchor=north]] {$P_\mathrm{eq}(r')$};
    \draw[graph_label] (11.8,-19) -- (11.8,-20.4) node[anchor=north] {$P_\mathrm{eq}(r)$};
    \draw[graph_label] (13,-19) -- (13,-19.2) node[anchor=north] {$P_\mathrm{sp}^\mathrm{cyl}(r')$};
    \draw[graph_label] (14.2,-19) -- (14.2,-20.4) node[anchor=north] {$P_\mathrm{sp}^\mathrm{cyl}(r)$};
    \draw[graph_label] (19,-19) -- (19,-19.2) node[anchor=north] {$P_\mathrm{sv}$};
    %y labels
    \draw[graph_label] (3,-19) -- (2.5,-19) node[anchor=east] {$0$};
    \draw[graph_label] (3,-14) -- (2.8,-14) node[anchor=east] {$1$};
    %grid
    \draw[dashed, graph_label] (3,-14.3) -- (13,-14.3);
    \draw[dashed, graph_label] (3,-14) -- (19,-14);
    \draw[dashed, graph_label] (19,-19) -- (19,-14);
    \draw[dashed, graph_label] (13,-19) -- (13,-18);
    \draw[dashed, graph_label] (3,-18.5) -- (13,-18.5);
    %isotherm
    \draw[line width = 0.1cm, color = yellow!90, directed] (3,-19) -- (9,-18.7);
    \draw[line width= 0.1cm, color = green!70, directed] (9,-18.7) -- (13,-18.5);
    \draw[line width = 0.1cm, color = red!70, directed] (13,-18.5) -- (13,-14.3);
    \draw[line width = 0.1cm, color = blue!70, directed] (13,-14.3) -- (19,-14);
    \end{tikzpicture}

\end{document}