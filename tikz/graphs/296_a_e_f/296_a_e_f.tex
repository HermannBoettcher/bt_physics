\documentclass[../../../thesis.tex]{subfiles}

\begin{document}
  \begin{figure}[ht]
		\centering
    \begin{tikzpicture}[
			spy using outlines={ellipse, magnification=4, connect spies}]
        \def\PrelMin{.86}
				\def\PrelMax{1}
				\def\TransMin{1e-8}
				\def\TransMax{1}
				\def\LfMin{0}
				\def\LfMax{2}
        %
        \begin{axis}[
          /tikz/line join=bevel,
          width=0.8*\textwidth,
          height=0.5*\textwidth,
          grid,
					axis y line*=left,
          legend style={at={(0,.5)}, legend columns=1, anchor=north west},
          every axis plot,
					%axis y line*=left,
					line width = 1pt,
					%	minor x tick num= ,
					%	minor y tick num= ,
					xmin = \PrelMin, xmax = \PrelMax,
					ymin = \LfMin, ymax = \LfMax,
					xlabel = {Relative pressure $P_\mathrm{rel}$},
					ylabel = {Liquid fraction $LF$},
					ytick = {0,0.25,0.50,0.75,1},
          ]
					% Add plots
					\addplot[mark=none, color=red] table [x=Prel,y=liquid_fraction]{tikz/graphs/296_a_e_f/296a_cond_2.txt};
					\addlegendentry{$LF_\mathrm{296a}$}
					\addplot[mark=none, color=blue] table [x=Prel,y=liquid_fraction]{tikz/graphs/296_a_e_f/296e_cond_2.txt};
					\addlegendentry{$LF_\mathrm{296e}$}
					\addplot[mark=none, color=green] table [x=Prel,y=liquid_fraction]{tikz/graphs/296_a_e_f/296f_cond_2.txt};
					\addlegendentry{$LF_\mathrm{296f}$}
					\addplot[mark=none, color=red!50] table [x=Prel,y=liquid_fraction]{tikz/graphs/296_a_e_f/296a_evap_2.txt};
					\addplot[mark=none, color=blue!50] table [x=Prel,y=liquid_fraction]{tikz/graphs/296_a_e_f/296e_evap_2.txt};
					\addplot[mark=none, color=green!50] table [x=Prel,y=liquid_fraction]{tikz/graphs/296_a_e_f/296f_evap_2.txt};
        \end{axis}
        \begin{axis}[
          /tikz/line join=bevel,
          width=0.8*\textwidth,
          height=0.5*\textwidth,
					ymode=log,
          ymajorgrids=true,
					%axis y line*=left,
          legend style={at={(0,.5)}, legend columns=1, anchor=south west},
          every axis plot,
					axis y line*=right,
					line width = 1pt,
					%	minor x tick num= ,
					%	minor y tick num= ,
					xmin = \PrelMin, xmax = \PrelMax,
					ymin = \TransMin, ymax = \TransMax,
					ylabel = {Transmission $T$},
					ytick = {1,1e-1,1e-2,1e-3},
          xtick = {},
          ]
					% Add plots
					\addplot[mark=none, color=red] table [x=Prel,y=transmission]{tikz/graphs/296_a_e_f/296a_cond_2.txt};
					\addlegendentry{$T_\mathrm{296a}$}
					\addplot[mark=none, color=blue] table [x=Prel,y=transmission]{tikz/graphs/296_a_e_f/296e_cond_2.txt};
					\addlegendentry{$T_\mathrm{296e}$}
					\addplot[mark=none, color=green] table [x=Prel,y=transmission]{tikz/graphs/296_a_e_f/296f_cond_2.txt};
					\addlegendentry{$T_\mathrm{296f}$}
					\addplot[mark=none, color=red!50] table [x=Prel,y=transmission]{tikz/graphs/296_a_e_f/296a_evap_2.txt};
					\addplot[mark=none, color=blue!50] table [x=Prel,y=transmission]{tikz/graphs/296_a_e_f/296e_evap_2.txt};
					\addplot[mark=none, color=green!50] table [x=Prel,y=transmission]{tikz/graphs/296_a_e_f/296f_evap_2.txt};
        \end{axis}
    \end{tikzpicture}
    \label{fig:full-comp-w296}
    \caption{Comparison of the isotherms measured for the membranes 296a, 296e and 296f. While the pore diameters are marginally smaller for the membranes 296e and 296f than they are for 296a, the overall picture is equivalent. It makes for closed pores with the same funnelling aspect.}
  \end{figure}
\end{document}
