\documentclass[../../../thesis.tex]{subfiles}

\begin{document}
  \begin{figure}[ht]
		\centering
    \begin{tikzpicture}[spy using outlines={ellipse, magnification=4, connect spies}]
        \def\PrelMin{.86}
				\def\PrelMax{1}
				\def\TransMin{1e-8}
				\def\TransMax{10}
				\def\LfMin{0}
				\def\LfMax{2}
        %
        \begin{axis}[
          /tikz/line join=bevel,
          width=0.8*\textwidth,
          height=0.5*\textwidth,
          grid,
					axis y line*=left,
          legend style={at={(0,0.5)}, legend columns=1, anchor=north west},
          every axis plot,
					axis y line*=left,
					line width = 1pt,
					%	minor x tick num= ,
					%	minor y tick num= ,
					xmin = \PrelMin, xmax = \PrelMax,
					ymin = \LfMin, ymax = \LfMax,
					xlabel = {Relative pressure $P_\mathrm{rel}$},
					ylabel = {Liquid fraction $LF$},
					ytick = {0,0.25,0.50,0.75,1},
          ]
					% Add plots
					\addplot[mark=none, color=red] table [x=Prel,y=liquid_fraction]{tikz/graphs/immersion_experiment/296a_cond_2.txt};
					\addlegendentry{$LF_{t_\mathrm{immerse}=\SI{0}{\minute}}$}
					\addplot[mark=none, color=purple] table [x=Prel,y=liquid_fraction]{tikz/graphs/immersion_experiment/296c_cond_2.txt};
					\addlegendentry{$LF_{t_\mathrm{immerse}=\SI{6,5}{\minute}}$}
					\addplot[mark=none, color=orange] table [x=Prel,y=liquid_fraction]{tikz/graphs/immersion_experiment/296d_cond_2.txt};
					\addlegendentry{$LF_{t_\mathrm{immerse}=\SI{13}{\minute}}$}
					\addplot[mark=none, color=red!50] table [x=Prel,y=liquid_fraction]{tikz/graphs/immersion_experiment/296a_evap_2.txt};
					\addplot[mark=none, color=purple!50] table [x=Prel,y=liquid_fraction]{tikz/graphs/immersion_experiment/296c_evap_2.txt};
					\addplot[mark=none, color=orange!50] table [x=Prel,y=liquid_fraction]{tikz/graphs/immersion_experiment/296d_evap_2.txt};
					\coordinate (spypoint1) at (axis cs:.9,0.16);
  				\coordinate (magnifyglass1) at (axis cs:.96,.3);
					\coordinate (spypoint2) at (axis cs:.9304,.95);
  				\coordinate (magnifyglass2) at (axis cs:.97,1.35);
        \end{axis}
        \spy[size=2.5cm, height=1.5cm] on (spypoint1) in node[fill=white] at (magnifyglass1);
        \spy[size=4.4cm, height=2cm] on (spypoint2) in node[fill=white] at (magnifyglass2);
				% transmission
				\begin{axis}[
          /tikz/line join=bevel,
          width=0.8*\textwidth,
          height=0.5*\textwidth,
					ymode=log,
					axis y line*=right,
					line width = 1pt,
					%	minor x tick num= ,
					%	minor y tick num= ,
					xmin = \PrelMin, xmax = \PrelMax,
					ymin = \TransMin, ymax = \TransMax,
					ylabel = {Transmission $T$},
					ytick = {1,1e-1,1e-2,1e-3},
          ymajorgrids=true,
          legend style={at={(0,0.5)}, legend columns=1, anchor=south west},
					%title={Membrane 296b - $\SI{60}{\micro\meter}$}
					]
					% Add plots
          \addplot[mark=none, color=red] table [x=Prel,y=transmission]{tikz/graphs/immersion_experiment/296a_cond_2.txt};
          \addlegendentry{$T_{t_\mathrm{immerse}=\SI{0}{\minute}}$}
          \addplot[mark=none, color=purple] table [x=Prel,y=transmission]{tikz/graphs/immersion_experiment/296c_cond_2.txt};
          \addlegendentry{$T_{t_\mathrm{immerse}=\SI{6,5}{\minute}}$}
          \addplot[mark=none, color=orange] table [x=Prel,y=transmission]{tikz/graphs/immersion_experiment/296d_cond_2.txt};
          \addlegendentry{$T_{t_\mathrm{immerse}=\SI{13}{\minute}}$}
          \addplot[mark=none, color=red!50] table [x=Prel,y=transmission]{tikz/graphs/immersion_experiment/296a_evap_2.txt};
          \addplot[mark=none, color=purple!50] table [x=Prel,y=transmission]{tikz/graphs/immersion_experiment/296c_evap_2.txt};
          \addplot[mark=none, color=orange!50] table [x=Prel,y=transmission]{tikz/graphs/immersion_experiment/296d_evap_2.txt};
				\end{axis}
    \end{tikzpicture}
    \label{fig:immersed-comp-w296}
    \caption{$P_\mathrm{rel}$ isotherm of membrane 292d. The transmission is not
		normalized but in arbitrary units plotted on a log scale. Please refer to
		\cref{subsec:light-transmission-interpretation} for the interpretation of the
		different magnitudes of the transmission drops for absorption and desorption.}
  \end{figure}
\end{document}
