\documentclass[../../../thesis.tex]{subfiles}

\begin{document}
  \begin{figure}[tpb]
		\centering
    \subfloat[]{
      \tikzsetnextfilename{295_f_200ALD_p}
      \begin{tikzpicture}
          \def\PrelMin{.55}
  				\def\PrelMax{.97}
  				\def\TransMin{1e-4}
  				\def\TransMax{1}
  				\def\LfMin{0}
  				\def\LfMax{2}
          %
          \begin{axis}[
            /tikz/line join=bevel,
            width=0.8*\textwidth,
            height=0.5*\textwidth,
            grid,
            legend style={at={(0,.5)}, legend columns=1, anchor=north west},
            every axis plot,
  					axis y line*=left,
  					line width = 1pt,
  					xmin = \PrelMin, xmax = \PrelMax,
  					ymin = \LfMin, ymax = \LfMax,
  					xlabel = {Relative pressure $P_\mathrm{rel}$},
  					ylabel = {Liquid fraction $LF$},
  					ytick = {0,0.25,0.50,0.75,1},
            ]
  					% Add plots
  					\addplot[mark=none, color=red] table [x=Prel,y=liquid_fraction]{tikz/graphs/295_e_f_ALD/295f200ALD_cond_1.txt};
  					\addlegendentry{$LF_\mathrm{cond}^\mathrm{295f''}$}
  					\addplot[mark=none, color=red!50] table [x=Prel,y=liquid_fraction]{tikz/graphs/295_e_f_ALD/295f200ALD_evap_1.txt};
  					\addlegendentry{$LF_\mathrm{evap}^\mathrm{295f''}$}
          \end{axis}
          \begin{axis}[
            /tikz/line join=bevel,
            width=0.8*\textwidth,
            height=0.5*\textwidth,
            legend style={at={(1,.5)}, legend columns=1, anchor=south east},
            every axis plot,
  					axis y line*=right,
  					line width = 1pt,
            ymode=log,
  					xmin = \PrelMin, xmax = \PrelMax,
  					ymin = \TransMin, ymax = \TransMax,
  					ylabel = {Transmission $T$},
            ymajorgrids=true,
  					ytick = {1,1e-1,1e-2},
            ]
  					% Add plots
  					\addplot[mark=none, color=red] table [x=Prel,y=transmission]{tikz/graphs/295_e_f_ALD/295f200ALD_cond_1.txt};
  					\addlegendentry{$T_\mathrm{cond}^\mathrm{295f''}$}
  					\addplot[mark=none, color=red!50] table [x=Prel,y=transmission]{tikz/graphs/295_e_f_ALD/295f200ALD_evap_1.txt};
  					\addlegendentry{$T_\mathrm{evap}^\mathrm{295f''}$}
          \end{axis}
      \end{tikzpicture}
      \label{fig:295-f-200ald}
    }
    \\
    \subfloat[]{
      \tikzsetnextfilename{295_f_200ALD_kelvin}
      \begin{tikzpicture}
          \def\rMin{0}
  				\def\rMax{50}
  				\def\TransMin{1e-4}
  				\def\TransMax{1}
  				\def\LfMin{0}
  				\def\LfMax{2}
          %
          \begin{axis}[
            /tikz/line join=bevel,
            width=0.8*\textwidth,
            height=0.5*\textwidth,
            grid,
            legend style={at={(1,.5)}, legend columns=1, anchor=north east},
            every axis plot,
  					axis y line*=left,
  					line width = 1pt,
  					xmin = \rMin, xmax = \rMax,
  					ymin = \LfMin, ymax = \LfMax,
  					xlabel = {Relative pressure $P_\mathrm{rel}$},
  					ylabel = {Liquid fraction $LF$},
  					ytick = {0,0.25,0.50,0.75,1},
            ]
  					% Add plots
  					\addplot[mark=none, color=blue] table [x=r_kelvin,y=liquid_fraction]{tikz/graphs/295_e_f_ALD/295f200ALD_cp_cond_1.txt};
  					\addlegendentry{$LF_\mathrm{cond,cp}^\mathrm{295f''}$}
  					\addplot[mark=none, color=red] table [x=r_kelvin,y=liquid_fraction]{tikz/graphs/295_e_f_ALD/295f200ALD_cond_1.txt};
  					\addlegendentry{$LF_\mathrm{cond,op}^\mathrm{295f''}$}
  					\addplot[mark=none, color=green!50] table [x=r_kelvin,y=liquid_fraction]{tikz/graphs/295_e_f_ALD/295f200ALD_evap_1.txt};
  					\addlegendentry{$LF_\mathrm{evap,cp/op}^\mathrm{295f''}$}
  					\coordinate (roi) at (axis cs:22,0.55);
          \end{axis}
          \draw (roi) ellipse (2cm and 1.5cm);
          \begin{axis}[
            /tikz/line join=bevel,
            width=0.8*\textwidth,
            height=0.5*\textwidth,
            legend style={at={(1,.5)}, legend columns=1, anchor=south east},
            every axis plot,
  					axis y line*=right,
  					line width = 1pt,
            ymode=log,
  					xmin = \rMin, xmax = \rMax,
  					ymin = \TransMin, ymax = \TransMax,
  					ylabel = {Transmission $T$},
            ymajorgrids=true,
  					ytick = {1,1e-1,1e-2},caption
            ]
  					% Add plots
  					\addplot[mark=none, color=blue] table [x=r_kelvin,y=transmission]{tikz/graphs/295_e_f_ALD/295f200ALD_cp_cond_1.txt};
  					\addlegendentry{$T_\mathrm{cond,cp}^\mathrm{295f''}$}
  					\addplot[mark=none, color=red] table [x=r_kelvin,y=transmission]{tikz/graphs/295_e_f_ALD/295e200ALD_cond_1.txt};
  					\addlegendentry{$T_\mathrm{cond,op}^\mathrm{295f''}$}
  					\addplot[mark=none, color=green!50] table [x=r_kelvin,y=transmission]{tikz/graphs/295_e_f_ALD/295f200ALD_evap_1.txt};
  					\addlegendentry{$T_\mathrm{evap,cp/op}^\mathrm{295f''}$}
  					\coordinate (roi) at (axis cs:17,0.45);
          \end{axis}
          \draw (roi) ellipse (1cm and 0.3cm);
      \end{tikzpicture}
      \label{fig:295-f-200ald-kelvin}
    }
    \caption{Plots for the analysis of membrane 295f''. \protect\subref{fig:295-f-200ald} shows the isotherms over relative pressure while \protect\subref{fig:295-f-200ald-kelvin} shows it over a \textsc{Kelvin} diameter axis. On the latter graph, the isotherm data of membrane 295f'' is converted once using the \textsc{Kelvin} equation for open pores and once using it for closed pores. The black circles mark the overlap of the signals on the same diameter range. }
    \label{fig:295f-200ald}
  \end{figure}
\end{document}
