\documentclass[../../../thesis.tex]{subfiles}

\begin{document}
  \begin{figure}[htpb]
		\centering
    \subfloat[]{
    \tikzsetnextfilename{295_296_funnelling_cp}
      \begin{tikzpicture}
          \def\rKelvinMin{30}
  				\def\rKelvinMax{80}
  				\def\TransMin{1e-4}
  				\def\TransMax{1}
  				\def\LfMin{0}
  				\def\LfMax{2}
          %
          \begin{axis}[
            /tikz/line join=bevel,
            width=0.8\textwidth,
            height=0.5*\textwidth,
            grid,
  					axis y line*=left,
            legend style={at={(0,0.5)}, legend columns=2, anchor=north west},
            every axis plot,
  					axis y line*=left,
  					line width = 1pt,
  					xmin = \rKelvinMin, xmax = \rKelvinMax,
  					ymin = \LfMin, ymax = \LfMax,
  					xlabel = {Pore diameter $d_\mathrm{pore}^\mathrm{Kelvin}$ in $\si{\nano\meter}$},
  					ylabel = {Liquid fraction $LF$},
  					ytick = {0,0.25,0.50,0.75,1},
            ]
  					% Add plots
  					\addplot[mark=none, color=red] table [x=r_kelvin,y=liquid_fraction]{tikz/graphs/295_296_funnelling/295a_cond_4.txt};
  					\addlegendentry{$LF_\mathrm{cond}^\mathrm{295a}$}
  					\addplot[mark=none, color=red!50] table [x=r_kelvin,y=liquid_fraction]{tikz/graphs/295_296_funnelling/295a_evap_4.txt};
  					\addlegendentry{$LF_\mathrm{evap}^\mathrm{295a}$}
  					\addplot[mark=none, color=blue] table [x=r_kelvin,y=liquid_fraction]{tikz/graphs/295_296_funnelling/296b_cp_cond_1.txt};
  					\addlegendentry{$LF_\mathrm{cond}^\mathrm{296b}$}
  					\addplot[mark=none, color=blue!50] table [x=r_kelvin,y=liquid_fraction]{tikz/graphs/295_296_funnelling/296b_cp_evap_1.txt};
  					\addlegendentry{$LF_\mathrm{evap}^\mathrm{296b}$}
            \addplot[color=magenta, mark=none, dashed]
            coordinates {
            (37,0)
            (37,2)};
            \addplot[color=magenta, mark=none, dashed]
            coordinates {
            (54,0)
            (54,2)};
            \addplot[color=orange, mark=none, dashed]
            coordinates {
            (48,0)
            (48,2)};
            \addplot[color=orange, mark=none, dashed]
            coordinates {
            (63,0)
            (63,2)};
          \end{axis}
  				% transmission
  				\begin{axis}[
            /tikz/line join=bevel,
            width=0.8*\textwidth,
            height=0.5*\textwidth,
  					ymode=log,
  					axis y line*=right,
  					line width = 1pt,
  					xmin = \rKelvinMin, xmax = \rKelvinMax,
  					ymin = \TransMin, ymax = \TransMax,
  					ylabel = {Transmission $T$},
  					ytick = {10,1,1e-1,1e-2},
            ymajorgrids=true,
            legend style={at={(0,0.5)}, legend columns=2, anchor=south west},
  					]
  					% Add plots
  					\addplot[mark=none, color=red] table [x=r_kelvin,y=transmission]{tikz/graphs/295_296_funnelling/295a_cond_4.txt};
  					\addlegendentry{$T_\mathrm{cond}^\mathrm{295a}$}
  					\addplot[mark=none, color=red!50] table [x=r_kelvin,y=transmission]{tikz/graphs/295_296_funnelling/295a_evap_4.txt};
  					\addlegendentry{$T_\mathrm{evap}^\mathrm{295a}$}
  					\addplot[mark=none, color=blue] table [x=r_kelvin,y=transmission]{tikz/graphs/295_296_funnelling/296b_cp_cond_1.txt};
  					\addlegendentry{$T_\mathrm{cond}^\mathrm{296b}$}
  					\addplot[mark=none, color=blue!50] table [x=r_kelvin,y=transmission]{tikz/graphs/295_296_funnelling/296b_cp_evap_1.txt};
  					\addlegendentry{$T_\mathrm{evap}^\mathrm{296b}$}
  				\end{axis}
      \end{tikzpicture}
      \label{fig:295-296-funnelling-cp}
    }
    \\
    \subfloat[]{
    \tikzsetnextfilename{295_296_funnelling_op}
      \begin{tikzpicture}
          \def\rKelvinMin{25}
  				\def\rKelvinMax{70}
  				\def\TransMin{1e-8}
  				\def\TransMax{1}
  				\def\LfMin{0}
  				\def\LfMax{2}
          %
          \begin{axis}[
            /tikz/line join=bevel,
            width=0.8*\textwidth,
            height=0.5*\textwidth,
            grid,
  					axis y line*=left,
            legend style={at={(0,0.5)}, legend columns=2, anchor=north west},
            every axis plot,
  					axis y line*=left,
  					line width = 1pt,
  					xmin = \rKelvinMin, xmax = \rKelvinMax,
  					ymin = \LfMin, ymax = \LfMax,
  					xlabel = {Pore diameter $d_\mathrm{pore}^\mathrm{Kelvin}$},
  					ylabel = {Liquid fraction $LF$},
  					ytick = {0,0.25,0.50,0.75,1},
            ]
  					% Add plots
  					\addplot[mark=none, color=red] table [x=r_kelvin,y=liquid_fraction]{tikz/graphs/295_296_funnelling/295g_cond_1.txt};
  					\addlegendentry{$LF_\mathrm{cond}^\mathrm{295g'}$}
  					\addplot[mark=none, color=red!50] table [x=r_kelvin,y=liquid_fraction]{tikz/graphs/295_296_funnelling/295g_evap_1.txt};
  					\addlegendentry{$LF_\mathrm{evap}^\mathrm{295g'}$}
  					\addplot[mark=none, color=blue] table [x=r_kelvin,y=liquid_fraction]{tikz/graphs/295_296_funnelling/296b_op_cond_1.txt};
  					\addlegendentry{$LF_\mathrm{cond}^\mathrm{296b'}$}
  					\addplot[mark=none, color=blue!50] table [x=r_kelvin,y=liquid_fraction]{tikz/graphs/295_296_funnelling/296b_op_evap_1.txt};
  					\addlegendentry{$LF_\mathrm{evap}^\mathrm{296b'}$}
            \addplot[color=magenta, mark=none, dashed]
            coordinates {
            (46,0)
            (46,2)};
            \addplot[color=magenta, mark=none, dashed]
            coordinates {
            (59,0)
            (59,2)};
            \addplot[color=orange, mark=none, dashed]
            coordinates {
            (56,0)
            (56,2)};
            \addplot[color=orange, mark=none, dashed]
            coordinates {
            (65,0)
            (65,2)};
          \end{axis}
  				% transmission
  				\begin{axis}[
            /tikz/line join=bevel,
            width=0.8*\textwidth,
            height=0.5*\textwidth,
  					ymode=log,
  					axis y line*=right,
  					line width = 1pt,
  					xmin = \rKelvinMin, xmax = \rKelvinMax,
  					ymin = \TransMin, ymax = \TransMax,
  					ylabel = {Transmission $T$},
  					ytick = {10,1,1e-1,1e-2,1e-3},
            ymajorgrids=true,
            legend style={at={(0,0.5)}, legend columns=2, anchor=south west},
  					]
  					% Add plots
  					\addplot[mark=none, color=red] table [x=r_kelvin,y=transmission]{tikz/graphs/295_296_funnelling/295g_cond_1.txt};
  					\addlegendentry{$T_\mathrm{cond}^\mathrm{295g'}$}
  					\addplot[mark=none, color=red!50] table [x=r_kelvin,y=transmission]{tikz/graphs/295_296_funnelling/295g_evap_1.txt};
  					\addlegendentry{$T_\mathrm{evap}^\mathrm{295g'}$}
  					\addplot[mark=none, color=blue] table [x=r_kelvin,y=transmission]{tikz/graphs/295_296_funnelling/296b_op_cond_1.txt};
  					\addlegendentry{$T_\mathrm{cond}^\mathrm{296b'}$}
  					\addplot[mark=none, color=blue!50] table [x=r_kelvin,y=transmission]{tikz/graphs/295_296_funnelling/296b_op_evap_1.txt};
  					\addlegendentry{$T_\mathrm{evap}^\mathrm{296b'}$}
  				\end{axis}
      \end{tikzpicture}
      \label{fig:295-296-funnelling-op}
    }
    \caption{Comparison of closed pore membranes \protect\subref{fig:295-296-funnelling-cp} and open pore membranes \protect\subref{fig:295-296-funnelling-op} of f different thickness. The membranes of wafer 295 are $l_\mathrm{pore}=\SI{30}{\micro\meter}$ thick, wheras the membranes of the wwafers 292 and 296 are $l_\mathrm{pore}=\SI{60}{\micro\meter}$ thick. For interpretations please refer to \cref{sec:thinner-membranes}.}
    \label{fig:205-296-funnelling}
  \end{figure}
\end{document}
