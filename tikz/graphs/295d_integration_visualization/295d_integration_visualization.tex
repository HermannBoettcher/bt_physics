\documentclass[../../../thesis.tex]{subfiles}

\begin{document}
  \begin{figure}[ht]
    \subfloat[]{
    \tikzsetnextfilename{integration_visualization}
      \begin{tikzpicture}
        \def\tMin{-1}
        \def\tMax{16}
        \def\PMin{-40]}
        \def\PMax{60}
        \def\LineWidth{1pt}
        %
        \begin{axis}[
          /tikz/line join=bevel,
          width=0.45*\textwidth,
          height=0.45*\textwidth,
          line width = 1pt,
          xmin = \tMin, xmax = \tMax,
          ymin = \PMin, ymax = \PMax,
          xlabel = {Normalized time $t$ in $\si\hour$},
          ylabel = {Time derivative $\dot{P}_\mathrm{cell}$ in $\si{\milli\bar\per\hour}$},
          grid,
          legend style={at={(1,1)}, legend columns=1, anchor=north east},
          every axis plot,
          ]
          % Add plots
          \addplot[mark=none, color=blue] table [x=t,y=interpol]{tikz/graphs/295d_integration_visualization/296b_op_cond_1.txt};
          \addlegendentry{interpolation}
          \addplot[mark=none, color=red] table [x=t,y=dPcell]{tikz/graphs/295d_integration_visualization/296b_op_evap_1.txt};
          \addlegendentry{membrane}
          \addplot[name path=interpolation, mark=none, color=green] table [x=t,y=delta_dPcell]{tikz/graphs/295d_integration_visualization/296b_op_cond_1.txt};
          \addlegendentry{substraction}
          \path[name path=xAxis] (\pgfkeysvalueof{/pgfplots/xmin},0) -- (\pgfkeysvalueof{/pgfplots/xmax},0);
          \addplot[green!30] fill between[of=interpolation and xAxis];
          \addplot[mark=none, color=blue] table [x=t,y=interpol]{tikz/graphs/295d_integration_visualization/296b_op_evap_1.txt};
          \addplot[mark=none, color=red] table [x=t,y=dPcell]{tikz/graphs/295d_integration_visualization/296b_op_evap_1.txt};
          \addplot[name path=interpolation, mark=none, color=green] table [x=t,y=delta_dPcell]{tikz/graphs/295d_integration_visualization/296b_op_evap_1.txt};
        \end{axis}
      \end{tikzpicture}
      \label{fig:iso-computation}
      }
    \hfill
    \subfloat[]{
    \tikzsetnextfilename{example_isotherm}
          \begin{tikzpicture}
            \def\tMin{0}
            \def\tMax{155}
            \def\PMin{0]}
            \def\PMax{17}
            \def\LineWidth{1pt}
            %
            \begin{axis}[
              /tikz/line join=bevel,
              width=0.45*\textwidth,
              height=0.45*\textwidth,
              %xmode=log,
              line width = 1pt,
              xmin = \tMin, xmax = \tMax,
              ymin = \PMin, ymax = \PMax,
              xlabel = {Pressure inside the cell $P_\mathrm{cell}$},
              ylabel = {Condensed amount of hexane $n$ in $\si{\micro\mol}$},
              %	legend entries={Avec,Sans},
              grid,
              legend style={at={(0,1)}, legend columns=1, anchor=north west},
              every axis plot,
              ]
              % Add plots
              \addplot[mark=none, color=blue] table [x=Pcell,y=mol]{tikz/graphs/295d_integration_visualization/296b_op_cond_1.txt};
              \addlegendentry{condensation}
              \addplot[mark=none, color=blue!50] table [x=Pcell,y=mol]{tikz/graphs/295d_integration_visualization/296b_op_evap_1.txt};
              \addlegendentry{evaporation}
            \end{axis}
          \end{tikzpicture}
          \label{fig:iso}
          }
          \caption{\protect\subref{fig:iso-computation} shows the raw data of the isotherm
          with membrane and interpolation of the reference isotherm without membrane.
          Also, the substraction of the latter (compare integrand of \cref{eq:ndot-membrane})
          is plotted where the area to be integrated for the absorption and desorption
          isotherm is shaded light green. The integration according to \cref{eq:nmembrane}
          results in the isotherm displayed in \protect\subref{fig:iso}.}
  \end{figure}

\end{document}
