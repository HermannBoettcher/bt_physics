\documentclass[../thesis.tex]{subfiles}

\begin{document}
    \section{Laplace-Young equation}
    \label{sec:laplace-young-equation}

        Due to the surface tension $\gamma$, the interface between two static fluids is curved. The shape of this curvature is related to the capillary pressure difference $\Delta P$ sustained across the mentioned interface by the \textsc{Young-Laplace} equation

        \begin{align}
            \begin{split}
                \Delta P &= -\gamma \cdot \nabla \uvec{n} \\
                &= \gamma \left( \frac{1}{R_1} + \frac{1}{R_2}\right),
                \label{eq:laplace-young}
            \end{split}
        \end{align}

        with the unit vector $\uvec{n}$ orthogonal to the interface and the curvature radii $R_1$ and $R_2$. The relation so implies, that the surface of a gas-liquid interface will always strive towards a homogeneous curvature to achieve an equilibrium state regarding pressure differences along the surface.
\end{document}
