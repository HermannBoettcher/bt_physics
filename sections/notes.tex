\documentclass[../thesis.tex]{subfiles}

\begin{document}
    \chapter{Notes}

        \section{26.04.2018}

            The phone calls with \textsc{Keller} and \textsc{Wika} yielded the following results:

            First, the \textsc{Keller} pressure gages use a ceramic membrane and thus another O-ring is needed to connect with the metal case. The latter has not been tested using Helium. It might be made of Viton, the correspondent was not sure though. Moreover, evacuating the gage with secondary vacuum might cause damage to the membrane, while evacuating the the surrounding of the gage might cause damage to the external electronics, as the cemented electronics might move and rip some contacts.

            Second, the \textsc{Wika} pressure gages use a membrane made of stainless steel (and some other alloys). The whole part that is in contact with the system is made of nothing but metal. Evacuating the gage itself is no problem at all, even using secondary vacuum. On the other hand, evacuating the surrounding of the gage might cause damage to the electronics as there are gore-tex membranes and other futile parts. Furthermore, welding underneath the hex (screw) should not cause any damage, infringes the warranty though. The sensible membrane is located approximately at the top of the hex. The hex itself \textit{should not} be heated though.


        \section{27.04.2018}

            Launched a \textit{rampe} to record six isotherms over the weekend. The opening of the \textsc{Pfeiffer} valve is set to $2,5$ for all cycles. Futhermore, the evaporation is done by pumping the Hexane from the system - not by impeding a greater temperature difference between tank and cell. Moreover, the condensation of bulk liquid shall be recorded in this time. Therefore, starting at $\SI{150}{\milli\bar}$, the \textsc{Pfeiffer} valve is set to stay open for five hours.


        \section{02.05.2018}
            The isotherms turned out nice up to this point. To be able to compare them to older ones recorded by Victor, the laser transmission measurement is to be set up for at least one more isotherm, as it is more reliable than the pressure loop isotherm.

            After the installation, the temperature $T_{cell}$ fluctuates on a scale of $\SI{0,4}{\celsius}$ and also, the heating power $P_{well}$ moves back and forth between minimum and maximum power. After experimenting and rebooting the respective programs, hard rebooting the hardware by unplugging does the trick.

            With the temperature regulation working the \textit{rampe} is restarted to do some more isotherms with the installed laser transmission measurement.


        \section{03.05.2018}
            To be able to evaluate the recorded isotherms another one without a sample inside the cell must be recorded as a \textit{reference isotherm}. After removing the sample the same \textit{rampe} as used for the isotherms with a sample inside the cell is started.


        \section{04.05.2018}
            Upon repacking the cell a cable of the temperature regulation has been unplugged yesterday. Therefore the temperature fluctuated a lot around $\SI{22}{\celsius}$. After fixing the problem, another rampe is started for the weekend in which for each temperature \SIlist{19;18;17}{\celsius} two isotherms are recorded.

        \section{09.05.2018}
            The last sample from the waver used previously is implanted into the cell. After, a series of isotherms is started again via a \textit{rampe}. This time, the opening of the \textsc{Pfeiffer} valve is changed: $U_{Vp} = \SIlist{2,3;2,5;2,7}{\volt}$.


        \section{15.05.2018}
            Yesterday night, another isotherm at a \textsc{Pfeiffer} opening of $U_{Vp} = \SI{2,5}{\volt}$ has een recorded. Now, the sample is changed for sample XXX (incbottle shape). Another rampe recording $U_{Vp} = \SI{2,5}{\volt}$ is started.
\end{document}
