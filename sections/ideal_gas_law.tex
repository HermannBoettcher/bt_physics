\documentclass[../thesis.tex]{subfiles}

\begin{document}
    \section{Ideal Gas Law}
    \label{sec:ideal-gas-law}

        The ideal gas law can be written as
        \begin{equation}
            PV = nRT
            \label{eq:ideal-gas-law}
        \end{equation}
        with the pressure $P$, volume $V$, number of moles of gas $n$, temperature $T$ and ideal gas constant
        \begin{equation}
            R=\SI{8.3144598(48)}{\joule\per\kelvin\per\mole}.
        \end{equation}
        It is an approximation of the behaviour of gases which can be applied for sufficiently low pressures and simple atoms or molecules which do not interact strongly (for example noble gases).


        \subsection{Boyle–Mariotte Law}
        \label{subsec:boyle-mariotte}

            For a constant temperature $T$ and constant number of moles of gas $n$,  the ideal gas law \cref{eq:ideal-gas-law} yields the \textsc{Boyle-Mariotte} law
            \begin{equation}
                P_1V_1 = P_2V_2.
            \end{equation}
            It permits to measure a volume $V_2$ by expanding the known volume $V_1$ of gas and monitoring pressures $P_1$ and $P_2$ before and after the expansion.

\end{document}
