\documentclass[../thesis.tex]{subfiles}

\begin{document}

    \section{Motivation}

      The fundamental idea is to test models of condensation and evaporation of liquids in confinement. To this end, a well characterized confinement is neccessary. The latter shall be realized using nanoporous alumina membranes. The production of these is rather advanced as they are commonly used for the production of nanowires. Moreover, the shape of the pores is idealy cylindrical which makes for a simple geometry. A nanoporous alumina membrane includes a large number of parallel pores. Therefore, any exeriment conducted on one of these membranes yields a result that averages over all of these pores. This is why, ideally, monodisperse alumina membranes shall be produced.

      As a liquid to condense inside the pores of these membranes, hexane is used. There are multiple reasons for that. First, other teams have used hexane for this type of experiments before. This means that reference isotherms exist that the results can be compared to. Moreover, hexane permits to experiment at ambient temperature which makes things much easier. The fact that this is only a preexperiment for later experiments using helium, the use of liquids like nitrogen seems to make experimenting more difficult while offering no advantages.

      The models to be tested are primarily the \textsc{Kelvin} equation and the \textsc{Saam} and \textsc{Cole} theory. There are multiple variations though, as for example there is a modified \textsc{Kelvin} equation taking into account the wetting film which the basic \textsc{Kelvin} equation does not. In addition to that, thermal activation and the temperature relative to the critical temperature of the used fluid are relevant. The latter is one reason why finally, the experiment is planned to be conducted using helium. Finally, there are other more sphisticated models that need to be tested. Due to the small variations that are expected, again, the probing of the latter requires helium. Last but not least, in order to also study cavitation in alumina membranes, inc bottle like cavities with constricted openings must be produced. As for both, the helium experiments and for the inc bottle openings, pores of sub ten nanometer diameter must be synthesized in a controlled way, the production process needs to be improved. This becomes the main goal of the conducted experiments described in this article.


    \section{Troubles}

      Early on, the conducted experiments prove the membranes to be not as ideal as was hoped for. The idea of the perfect cylindrical pores of the same diameter is replaced by the conclusion that the pores' diameter is not only distributed over the membranes but also varies within a single pore. This variation manifests in the way of a funnelling aspect that makes for a conical shape, rather than a cylindrical, and also as corregations. As explained above, the confinement needs to be well characterized to be able to test the models of condensation and evaporation. In contrast to that, the geometrical shape of the pores does not allow for this kind of precise characterization. Using electron beam microscopy, the opening diameters of the pores can be checked on top and bottom side. Furthermore, a cross section view yields some information about the straightness and possible defects along the length of the pores. As the pores' length ranges in the oder of micrometers though, while the diameters' magnitude is nanometers, this still gives little information about the overall pore shape. Moreover, the microscopy images can not be precisely evaluated due to the unknown porosity of the membranes and again the lack of information on the shape. In the course of the experiments a way is developed to at least delimit a range for the porosity and so allow for a more sophisticated analysis of the electron beam microscopy images under the assumption of a certain pore geometry.

      The bottom line is that neither the exact geometry and distribution of pore shapes and sizes is known, nor are the condensation and evaporation models clear. Thus, the first step must be to understand and refine the production process of the membranes and to hereby make for the best possible characterization of the confinement.


    \section{Approach}

      To understand the membranes' characteristics and to use this knowledge to improve the production process, rather large pore diameters of above fourty nanometer diameter, for which \textsc{Kelvin} equation is assumed to be sufficiently precise, are probed using multiple techniques. The core of the conducted experiments relies on thermodynamics using volumetric measurements. In addition, optical measurements and also visual documentation are used along with electron beam microscopy. All the aquired data is then interpreted and tested for coherence to finally draw conclusions on the membranes' characteristics. From the aquired results, questions rise about the effects of different production steps which are then taken a closer look at.


\end{document}
