\documentclass[../thesis.tex]{subfiles}

\begin{document}
    \chapter{Introduction}

    \section{Initial Goal of the Internship}

        Generally speaking, the goal of the internship is testing condensation and evaporation theory in confinement. The realized confinement consists of nanoporous alumina membranes absorbing and desorbing hexane while the laws to be tested are primarily the \textsc{Kelvin} law and the \textsc{Saam} and \textsc{Cole} law. Furthermore, the membrane production shall be refined to the aim of creating cylindrical pores and also combining pores of different sizes to form inc bottle like confinements. The latter are to be used for studying cavitation in confinement.


    \section{Problem}

        The membranes are not as good as expected in the sense, that they are not of perfect cylindrical shape and are hard to be characterized. Therefore, neither the result of the condensation and evaporation theory nor the actual shape of the pores are known. Thus, the first step must be to understand and refine the production process of the membranes. To this end, MEB views ??? are used to probe the diameter of the created pores. As the pores also show a funneling aspect and corrugations, the porosity of the membranes is also not clear. This makes the MEB view analysis less reliable to random. A way must be found to measure the porosity of the membranes and to estimate the funneling aspect
\end{document}
