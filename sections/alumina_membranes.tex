\documentclass[../thesis.tex]{subfiles}

\begin{document}

    \section{Alumina Membranes}
    \label{sec:alumina-membranes}


        \subsection{Membrane Production}
        \label{subsec:membrane-production}

            The alumina membrane production process starts with a circular wafer of amorphous aluminum of $\SI{99,999}{\percent}$ purity. With an initial thickness of $h_\mathrm{\ce{Al}}=\SI{1}{\milli\meter}$ and diameter of $d_\mathrm{wafer}=\SI{5}{\centi\meter}$, one wafer is cut into twelve square membranes of side length $l_\mathrm{membrane}=\SI{1}{\centi\meter}$.

            First, the aluminum is anodized to create parallel pores that are arranged in a hexagonal order (compare section \cref{sssec:anodizing}). For the conducted experiment, pore lengths $l_\mathrm{pore}=\SIlist{30;60}{\micro\meter}$ are used. After the anodizing, the remaining aluminum of the wafer is dissolved by immersion in as specified in \cref{eq:aluminum-dis-acid} in \cref{sssec:al-dissolution}). Last, the so called \textit{barrier layer} closing the pores on the bottom end is etched using oxalic acid (section \cref{sssec:barrier-layer-dissolution}).


            \subsubsection{Anodizing}
            \label{sssec:anodizing}

                \Cref{fig:anodizing} shows a sketch of the anodizing setup. The bulk aluminum wafer (compare \cref{fig:bulk-al}), which functions as the anode, is glued to a copper plate using none conductive silver paste. Only one circular surface is exposed to the acid. The cathode is given by a platinum plate that is placed at a horizontal distance of $\SI{3}{\centi\meter}$ from the anode. The whole setup is immersed in oxalic acid that is stirred at all times. The anodizing of the wafers is conducted at a constant voltage $U_\mathrm{anodizing}$. To produce parallel pores, the anodizing is carried out in two stages. Both use the same setup with the same parameters.

                \subfile{tikz/membrane_production/anodizing.tex}


                The pore diameter $d_\mathrm{pore}$ and the inter pore distance $d_\mathrm{interpore}$ depends on the anodizing conditions, in explanation the voltage $U_\mathrm{anodizing}$, the oxalic acid's molar concentration $n_\mathrm{\ce{C2H2O4}}$ and the acid temperature $T_\mathrm{\ce{C2H2O4}}$. Feasible are diameters of $\SIrange{10}{100}{\nano\meter}$. To produce wafers with the pore specifications
                \begin{align}
                    \begin{split}
                        d_\mathrm{pore}=\SI{40}{\nano\meter},    \\
                        d_\mathrm{interpore}=\SI{100}{\nano\meter},
                    \end{split}
                    \label{eq:pore-specs}
                \end{align}
                the parameters
                \begin{align}
                    \begin{split}
                        U_\mathrm{anodizing}^{\SI{40}{\nano\meter}} &= \SI{40}{\volt}, \\
                        n_\mathrm{\ce{C2H2O4}} &= \SI{0,5}{\mol\per\litre},  \\
                        T_\mathrm{\ce{C2H2O4}} &= \SI{15}{\celsius}
                    \end{split}
                    \label{eq:anodizing-parameters}
                \end{align}
                are used for the two anodizings. This makes for a growth rate of alumina of
                \begin{equation}
                    r_{\ce{Al2O3}} \approx \SI{8}{\micro\meter\per\second}.
                \end{equation}
                \medskip

                The first anodizing treats the raw bulk aluminum wafer. By the oxalic acid \ce{C2H2O4}, pathways are etched into the aluminum which follow no pattern at first. Only some of them continue to grow forming real pores, though. With the ongoing anodizing pores start to combine to form larger pores and finally areas of hexagonally arrange pores are created throughout the wafer. The longer the anodizing process is carried out, the larger these ordered areas become. Figure \cref{fig:first-anodizing} shows the cross-section of a wafer observable at this point. Because the process transforms aluminum to alumina (compare section ???) at a given penetration speed, the so called \textit{barrier layer} of alumina separates the pores from the bulk aluminum.
                \medskip

                Before the second anodizing, the wafer is immersed in a mixture of chromic and phosphoric acid with the concentrations
                \begin{align}
                    \begin{split}
                        n_\mathrm{\ce{H3PO4}} &= \SI{0,4}{\mol\per\litre}    \\
                        n_\mathrm{\ce{C2O3}} &= \SI{0,2}{\mol\per\litre}
                    \end{split}
                \end{align}
                at at temperature of
                \begin{equation}
                    T_\mathrm{\ce{C2O3}}^\mathrm{\ce{H3PO4}}=\SI{0,2}{\mol\per\litre}
                \end{equation}
                Hereby the created alumina is dissolved. This yields a slightly thinner wafer of bulk aluminum, where the thickness $d_\mathrm{Al}'$ depends on the time of the first anodizing. The relevant difference to the initial wafer are the areas of hexagonally arranged hollows (compare figure \cref{fig:al-dissolution-1}).

                Due to this new surface structure, the second anodizing yields a wafer with a top layer of alumina penetrated by parallel, hexagonally arranged pores. Again, the \textit{barrier layer} of alumina separates the pores from the bulk aluminum as illustrated in figure \cref{fig:second-anodizing}. Increasing the time of the second anodizing increases the length of the pores $l_\mathrm{pore}$ and hereby the thickness of the final wafer. As the second anodizing is carried out under the same conditions as the first, the pore size $d_\mathrm{pore}$ is the same in both production steps.


                \subfile{tikz/membrane_production/production_stages.tex}


            \subsubsection{Aluminum Dissolution}
            \label{sssec:al-dissolution}

                To dissolve the remaining aluminum on the bottom side of the wafer, it is immersed in an acid composed of
                \begin{equation}
                    \SI{27,2}{\gram}\ce{CuCl2} + \SI{0,2}{\l}\ce{HCL}(\SI{37}{\percent}) + \SI{0,8}{\l}\ce{H2O}
                    \label{eq:aluminum-dis-acid}
                \end{equation}
                at a temperature of
                \begin{equation}
                    T = \SI{0}{\celsius}
                \end{equation}
                as shown in figure \cref{fig:al-dissolution-setup}. The latter transforms the aluminum to copper in a very exothermic reaction (compare section ???). A lot of turbulences are created and thus this step limits either the thickness or the size of the wafers because it can easily break.

                \subfile{tikz/membrane_production/aluminum_dissolution.tex}


            \subsubsection{Barrier Layer Dissolution}
            \label{sssec:barrier-layer-dissolution}

                The \textit{barrier layer} dissolution is one step of the membrane production chain that is taken a closer look at in this report. The procedure is done using phosphoric acid of the concentration
                \begin{equation}
                    n_\mathrm{\ce{H3O4P}} = \SI{0,2}{\mol\per\l}.
                \end{equation}
                To dissolve the \textit{barrier layer} and hereby open the pores, the wafers are floated on phosphoric acid as shown in figure \cref{fig:float}. Moreover, some membranes are immersed (compare \cref{fig:immerse}) in the run of the experiments with the same aim of dissolving the \textit{barrier layer}. If not explicitely mentioned though, the dissolution proccess has been conducted by floating for the respective membrane.

                \subfile{tikz/membrane_production/barrier_layer_dissolution.tex}


        \subsection{Membrane Specifications}

            For the experiments, square alumina membranes of
            DRAW A NICE IMAGE OF A MEMBRANE AND A PORE HERE???


\end{document}
